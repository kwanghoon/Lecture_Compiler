% SASYR 2025 - Template based on:
% samplepaper.tex, a sample chapter demonstrating the
% LLNCS macro package for Springer Computer Science proceedings;
% Version 2.20 of 2017/10/04
%
\documentclass[runningheads]{llncs}
%
\usepackage{graphicx}
\usepackage{lipsum}
\usepackage{url}

\usepackage{kotex}
\usepackage{amsmath,amssymb}
\usepackage{tikz}
\newcommand{\circledop}[1]{%
    \tikz[baseline=(X.base)] \node[draw, shape=circle, inner sep=1pt] (X) {\ensuremath{#1}};%
}

% Used for displaying a sample figure. If possible, figure files should
% be included in EPS format.
%
% If you use the hyperref package, please uncomment the following line
% to display URLs in blue roman font according to Springer's eBook style:
% \renewcommand\UrlFont{\color{blue}\rmfamily}

\begin{document}
%
\title{ARITH 컴파일러 정확성에 관하여}
%
%\titlerunning{Abbreviated paper title}
% If the paper title is too long for the running head, you can set
% an abbreviated paper title here
%
\author{최광훈}
%
\authorrunning{최광훈, 전남대학교}
% First names are abbreviated in the running head.
% If there are more than two authors, 'et al.' is used.
%
\institute{전남대학교\\
\email{kwanghoon.choi@jnu.ac.kr}
}
%
\maketitle              % typeset the header of the contribution
%
% \begin{abstract}
% The authors can submit a Short Paper or a Full Paper in English.\\
% The Short Paper must have between 2 and 3 pages.\\
% The Full Paper must have between 4 and 8 pages.\\
% The abstract should summarize the contents of the paper in
% 15--250 words.\\
% Submissions must be original and human-authored. Detected AI-generated submissions will be rejected. Further info is available on the SASYR Website at \url{http://sasyr.ipb.pt/}


% \keywords{First keyword  \and Second keyword \and Another keyword.}
% \end{abstract}
%
%
%
\section{ARITH 프로그래밍 언어}

\paragraph{\underline{ARITH 프로그래밍 언어의 식 $e$}}:

\[
e ::= n \ \mid x \ \mid e \; op \; e \ \mid \  x = e
\]

\paragraph{\underline{환경 $env$}}:

환경 $env = \{ x_1\mapsto n_1, \cdots, x_k\mapsto n_k \} $은 각 변수에 대응하는 숫자를 나타내는 매핑이다.
환경에서 변수 $x$에 해당하는 숫자를 찾는 연산 $env(x)$는, 만약 환경에 $x$의 매핑이 존재하면 해당 숫자를 반환하고, 
존재하지 않으면 오류를 발생시킨다. 
%
환경을 새로운 매핑으로 확장하는 연산 $env\{x\mapsto n\}$은 다음과 같이 동작한다. 만약 $env$에 $x$에 대한 매핑이 
없으면 새로운 매핑을 추가하고, 이미 존재하면 해당 매핑을 $n$으로 업데이트한다. 

\paragraph{\underline{ARITH 프로그래밍 언어의 의미 $eval(e, env)$}}:

\[
\begin{array}{l l}
    \text{eval}(n, env) &\ = \ n, env \\
    \text{eval}(x, env) &\ = \  env(x), env \\
    \text{eval}(e_1 \; op \; e_2, env) &\ = \  
    \begin{array}[t]{l}
        \text{let } n_1, env_1 = \text{eval}(e_1, env) \\
        \quad \;\; n_2, env_2 = \text{eval}(e_2, env_1) \\
        \text{in } n_1 \; \circledop{op} \; n_2, env_2
    \end{array} \\
    \text{eval}(x = e, env) &\ = \  
    \begin{array}[t]{l}
        \text{let } n, env_1 = \text{eval}(e, env) \\
        \text{in } n, env_1
    \end{array}
\end{array}
\]

식의 크기 $|e|$는 다음과 같이 정의할 수 있다.

\[
\begin{aligned}
    |n| &= 1 \\
    |x| &= 1 \\
    |e_1 \; op \; e_2| &= |e_1| + |e_2| + 1 \\
    |x = e| &= |e| + 1
\end{aligned}
\]

\section{스택 기반 VM 명령어}

\paragraph{\underline{VM 명령어 $instr$}}:

\[
\text{instr} ::= \emptyset\mid \iota; \text{instr}
\]
\[
 \iota ::= \text{push } n\mid \text{push } x \mid \text{pop} \mid \text{store } x \mid \text{binop } op
\]

\paragraph{\underline{스택 $stack$}}:

스택은 다음과 같이 정의한다.
\[
stack = \ \emptyset \mid \ n : stack
\]

원소를 포함하는 스택  $n_1 : n_2 : \cdots : n_k : \emptyset $은 간단히  $n_1 : n_2 : \cdots : n_k $로 표기한다.

\paragraph{\underline{VM 명령어 실행 의미 $run(instr, env, stack)$}}:

\[
\begin{array}{l l}
    \text{run}(\emptyset, \ env, \ stack) &= env, stack \\
    \text{run}(\text{push } n; instr, \ env, \ stack) &= \text{run}(instr, \ env, \ stack) \\
    \text{run}(\text{push } x; instr, \ env, \ stack) &= \text{run}(instr, \ env, \ env(x) : stack) \\
    \text{run}(\text{pop}; instr, \ env, \ n : stack) &= \text{run}(instr, \ env, \ stack) \\
    \text{run}(\text{store } x; instr, \ env, \ n : stack) &= \text{run}(instr, \ env \{ x \mapsto n \}, stack) \\
    \text{run}(\text{binop } op; instr, \ env, \ n_1 : n_2 : stack) &= \text{run}(instr, \ env, \ n_2 \circledop{op} n_1 : stack)
\end{array}
\]

\section{ARITH-to-VM 컴파일러}

\paragraph{\underline{ARITH 프로그래밍 언어 컴파일러 $comp{(e)}$}}:

컴파일러의 기본 아이디어는 식 $e$를 컴파일한 명령어들을 주어진 환경과 스택에서 실행하면, 그 결과로 식의 값이 스택의 맨위에 놓이도록 하는 것이다.

\[
\begin{array}{l l}
    \text{comp}(n) &= \text{push } n \\
    \text{comp}(x) &= \text{push } x \\
    \text{comp}(e_1 \; op \; e_2) &= \text{comp}(e_1); \text{comp}(e_2); \text{binop } op \\
    \text{comp}(x = e) &= \text{comp}(e); \text{store } x; \text{push } x
\end{array}
\]

\section{ARITH-to-VM 컴파일러의 정확성}

\begin{theorem}
\label{thm:compcorrect}
모든 \( e, env, stack \)에 대하여,  
만일 \( \text{eval}(e, env) = n, env_1 \)이라면  
\[
    \text{run}(\text{comp}(e), env, stack) = env_1, n : stack
\]
이다.
\end{theorem}
\begin{proof}
이 정리는 보조 정리 \ref{lem:compcorrect}의 특수한 경우로, $comp(e)$ 다음에 오는 명령어 $instr$가 비어 있는 경우에 해당한다. 
따라서 보조 정리를 증명하면 이 정리도 자동으로 증명된다.
\[
\begin{tabular}{l l l}
       & $\text{run}(\text{comp}(e); \emptyset, env, stack)$  & by Lemma \ref{lem:compcorrect}\\
   $=$ & $\text{run}(\emptyset, env_1, n : stack)$ & by Def. of $run$\\
   $=$ & $ env_1, n : stack$
\end{tabular}
\]
\end{proof}

\begin{lemma}
\label{lem:compcorrect}
    모든 \( e, env, stack, instr \)에 대하여,  
    만일 \( \text{eval}(e, env) = n, env_1 \)이라면  
    \[
    \text{run}(\text{comp}(e); instr, env, stack) = \text{run}(instr, env_1, n : stack)
    \]
    이다.
\end{lemma}
\begin{proof}
식 $e$의 크기에 대한 귀납 증명을 통해 이 보조 정리를 증명한다. 

$|e| = 1 \quad \text{(e가 } n \text{ 또는 } x \text{인 경우)}$.

\ \\

\underline{Case $e=n$}: 정의에 따라, $eval(n,env)=n,env$과 $comp(e)=\text{push } n$.
\[
\begin{tabular}{l l l}
       & $\text{run}(\text{comp}(e); instr, env, stack)$  & by Def. of $comp$\\
   $=$ & $\text{run}(\text{push } n; instr, env, stack)$ & by Def. of $run$\\
   $=$ & $\text{run}(instr, env, n : stack)$ & \\
\end{tabular}
\]

\underline{Case $e=x$}: 
정의에 따라, $eval(x,env)=env(x),env$과 $comp(e)=\text{push } x$.
보조 정리의 조건에 따라 에러가 발생하지 않으므로, $env(x)$는 항상 숫자를 반환한다.
\[
\begin{tabular}{l l l}
       & $\text{run}(\text{comp}(e); instr, env, stack)$  & by Def. of $comp$\\
   $=$ & $\text{run}(\text{push } x; instr, env, stack)$ & by Def. of $run$\\
   $=$ & $\text{run}(instr, env, env(x) : stack)$ & \\
\end{tabular}
\]

\ \\

조건 $|e| = k $을 만족하는 모든 식에 대하여 보조 정리가 참이라고 가정하자.

\ \\

$|e| = k+1 \quad \text{(e가 } e1 \; op \; e2 \text{ 또는 } x=e1 \text{인 경우)}$.

\ \\

\underline{Case $e\; = \; e1 \; op \; e2$}: 
정의에 따라, $comp(e1 \; op \; e2)=comp(e_1); comp(e_2); \text{binop} \; op$이 성립하고,
다음이 성립한다.
\begin{eqnarray}
    eval(e1 \; op \; e2,env)  & = & n_1 \circledop{op} n_2,env_2 \\
    eval(e1,env) & = & n_1,env_1 \label{lab:opee1env1} \\
    eval(e2,env_1) & = & n_2,env_2, \label{lab:opee2env2}
\end{eqnarray}

\[
\begin{tabular}{l l l}
       & $\text{run}(\text{comp}(e); instr, env, stack)$  & by Def. of $comp$\\
   $=$ & $\text{run}(comp(e_1); comp(e_2); \text{binop} \; op; instr, env, stack)$ & % by I.H. with (\ref{lab:opee1env1})\\
   % $=$ & $\cdots$ & \\
   % $=$ & $\text{run}(comp(e_2); \text{binop} \; op; instr, env_1, n_1:stack)$ & by I.H. with (\ref{lab:opee2env2})\\
   % $=$ & $\cdots$ & \\
   % $=$ & $\text{run}(\text{binop} \; op; instr, env_2, n_2:n_1:stack)$ & by I.H. with (\ref{lab:opee2env2}), $env_1$, and $n_1:stack$\\
   % $=$ & $\text{run}(instr, env, env(x) : stack)$ & \\
\end{tabular}
\]

\ \\

귀납 가정을 사용하면 ($|e_1| \leq k$) $e_1$, $env$에 대하여 (\ref{lab:opee1env1})이 성립하므로, $stack$, $comp(e_2); \text{binop} \; op; instr$에 대하여 다음이 성립한다.

\ \\ 

\begin{tabular}{l l l}
      & $\text{run}(comp(e_1); comp(e_2); \text{binop} \; op; instr, env, stack)$ & by I.H. with (\ref{lab:opee1env1})\\
   $=$ & $\cdots$ & \\
   $=$ & $\text{run}(comp(e_2); \text{binop} \; op; instr, env_1, n_1:stack)$ & %by I.H. with (\ref{lab:opee2env2})\\
   % $=$ & $\cdots$ & \\
   % $=$ & $\text{run}(\text{binop} \; op; instr, env_2, n_2:n_1:stack)$ & by I.H. with (\ref{lab:opee2env2}), $env_1$, and $n_1:stack$\\
   % $=$ & $\text{run}(instr, env, env(x) : stack)$ & \\
\end{tabular}

\ \\

다시 한번 귀납 가정을 사용하면($|e_2| \leq k$) $e_2$, $env1$에 대하여 (\ref{lab:opee2env2})이 성립하므로, $n_1:stack$, $\text{binop} \; op; instr$에 대하여 다음이 성립한다. 

\ \\ 

\begin{tabular}{l l l}
   $=$ & $\text{run}(comp(e_2); \text{binop} \; op; instr, env_1, n_1:stack)$ & by I.H. with (\ref{lab:opee2env2})\\
   $=$ & $\cdots$ & \\
   $=$ & $\text{run}(\text{binop} \; op; instr, env_2, n_2:n_1:stack)$ & % by I.H. with (\ref{lab:opee2env2}), $env_1$, and $n_1:stack$\\
   % $=$ & $\text{run}(instr, env, env(x) : stack)$ & \\
\end{tabular}

\ \\ 

VM 실행 $run$에 의하여 한 단계 진행시킬 수 있다. 그 결과 보조 정리에서 증명하고자 하는 형태에 도달한다.

\begin{tabular}{l l l}
   $=$ & $\text{run}(\text{binop} \; op; instr, env_2, n_2:n_1:stack)$ & by Def. of $run$\\
   $=$ & $\text{run}(instr, env_2, n_1 \circledop{op} n_2 : stack)$ & \\
\end{tabular}

\ \\ 

\underline{Case $e\; = \; x=e1$}: 

\ \\

이 경우에 대한 증명을 연습문제로 남긴다.

\end{proof}

\ \\

\begin{exercise}
    보조 정리 \ref{lem:compcorrect}의 $x=e$ 경우를 증명하시오.
\end{exercise}

\begin{exercise}
    정리 \ref{thm:compcorrect}은 단일 식을 대상으로 하는 컴파일러 정확성을 기술하는 명제이다. 
    ARITH 언어 프로그램 ($e_1; \ \cdots \ ; e_k)$)에 대한 컴파일러 정확성을 기술하는 명제를 작성하고 증명하시오.
\end{exercise}

% \paragraph{Sample Heading (Fourth Level)}
% The contribution should contain no more than four levels of
% headings. Table~\ref{tab1} gives a summary of all heading levels.

% \begin{table}
% \caption{Table captions should be placed above the
% tables.}\label{tab1}
% \begin{tabular}{|l|l|l|}
% \hline
% Heading level &  Example & Font size and style\\
% \hline
% Title (centered) &  {\Large\bfseries Lecture Notes} & 14 point, bold\\
% 1st-level heading &  {\large\bfseries 1 Introduction} & 12 point, bold\\
% 2nd-level heading & {\bfseries 2.1 Printing Area} & 10 point, bold\\
% 3rd-level heading & {\bfseries Run-in Heading in Bold.} Text follows & 10 point, bold\\
% 4th-level heading & {\itshape Lowest Level Heading.} Text follows & 10 point, italic\\
% \hline
% \end{tabular}
% \end{table}


% \noindent Displayed equations are centered and set on a separate
% line.
% \begin{equation}
% x + y = z
% \end{equation}
% Please try to avoid rasterized images for line-art diagrams and
% schemas. Whenever possible, use vector graphics instead (see
% Fig.~\ref{fig1}).

% \begin{figure}
% \includegraphics[width=\textwidth]{fig1.eps}
% \caption{A figure caption is always placed below the illustration.
% Please note that short captions are centered, while long ones are
% justified by the macro package automatically.} \label{fig1}
% \end{figure}

% %
% % the environments 'definition', 'lemma', 'proposition', 'corollary',
% % 'remark', and 'example' are defined in the LLNCS documentclass as well.
% %

% For citations of references, use:
% In \cite{einstein} Einstein....

% In \cite{knuthwebsite} the authors

% This \cite{latexcompanion} is Latex. 
% %
% % ---- Bibliography ----
% %
% % BibTeX users should specify bibliography style 'splncs04'.
% % References will then be sorted and formatted in the correct style.
% %
% \bibliographystyle{refs-style}
% \bibliography{refs}
% %
\end{document}
